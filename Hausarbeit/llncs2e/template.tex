

% This is based on the LLNCS.DEM the demonstration file of
% the LaTeX macro package from Springer-Verlag
% for Lecture Notes in Computer Science,
% version 2.4 for LaTeX2e as of 16. April 2010
%
% See http://www.springer.com/computer/lncs/lncs+authors?SGWID=0-40209-0-0-0
% for the full guidelines.
%
\documentclass{llncs}
\usepackage[utf8]{inputenc}
\usepackage[ngerman]{babel}
\begin{document}
\title{HTTP und HTTP/2}
\author{Michael Hochleitner}
\institute{Ludwig-Maximilians-Universit\"at M\"unchen}

\maketitle 
\begin{center}
Bachelorseminar ``Web Technologies'' \\
Wintersemester 2017/18
\end{center}

\begin{abstract}
HTTP ist ein Netzwerkprotokoll, das benutzt wird um Datenpakete zwischen Rechnern in Rechnernetzen auszutauschen. Die erste Version von HTTP wurde 1990 entwickelt. Seitdem wurde das Protokoll in mehreren Protokollversionen erweitert. Die Schwerpunkte bei den Erweiterungen des Protokolls sind Flexibilität und Performanz. Flexibilität bedeutet hier, dass das Protokoll erweitert werden kann. Es können neue Anfragemethoden, Errorcodes und Header hinzugefügt werden. HTTP baut auf einem Transportprotokoll auf. Verbesserungen in der Performanz erreicht man durch effiziente Nutzung des Transportprotokolls. Das wird in HTTP/2 durch Multiplexing umgesetzt.
\keywords{Protokoll, Rechnernetze, Internet, Hypertext, HTTP, HTTP/2}
\end{abstract}

\section{Einleitung}

Der Informatiker Tim Berners-Lee arbeitete in den 90er Jahren in der Forschungseinrichtung CERN in Genf. Dort arbeiteten viele Leute, die verschiedene Computer und verschiedene Textverarbeitungsprogramme benutzten, um ihre Ergebnisse zu dokumentieren. Der Austausch von digitalen Dokumenten war schwierig. Deshalb entwickelte er ein Dokumentationssystem. \begin{quote}What I was looking for fell under the general category of documentation systems-software that allows documents to be stored and later retrieved. \cite{Berners-Lee1999} \end{quote}
Das Dokumentationssystem kombinierte die Technologien Internet, Hypertext und das Aufrufen von Prozeduren auf anderen Computern, genannt RPC (Remote Procedure Call). Es wurde bekannt unter dem Namen World Wide Web.\newline Die Architektur des World Wide Web ist Client-Server. Der Client sollte ein Programm sein, mit dem man Hypertextseiten erstellen, durchblättern und editieren kann.\cite{Berners-Lee1999} Der Server sollte ein Programm sein, das die Hypertextseiten hält und den Benutzern des World Wide Web Zugriff darauf erlaubt.\cite{Berners-Lee1999} 
Durch das gemeinsame Dateiformat HTML (Hypertext Markup Language), die Verbindung von Computern über das Internet und das Abspeichern der HTML-Dateien auf einem Server wurde es möglich im World Wide Web auf digitale Dokumente zuzugreifen, die nicht dauerhaft auf dem Computer gepeichert sein müssen, auf dem man sie liest. Somit konnte man von Computern mit verschiedenen Betriebssystemen auf einen gemeinsamen Speicherort für Dokumente zugreifen.\newline Um Dateien auf einem Server für andere Computer zur Verfügung zu stellen, versendet man die Dateien in Datenpaketen. Dafür benötigt man ein Protokoll. Das Protokoll des World Wide Web ist HTTP, das Hypertext Transfer Protocol.
\section{HTTP Grundlagen}
HTTP wurde entwickelt, um das Internet und HTML kompatibel zu machen. HTTP gibt es in vier Versionen. Die Versionsnummern sind 0.9, 1.0 1.1 und 2. Definiert wird HTTP in Requests for Comments, die von der Internet Engineering Task Force (IETF) und der Internet Society (ISOC) herausgegeben werden. Die Versionsnummer 0.9 ist keine offizielle Versionsnummer, sondern eine Versionsnummer, die im Nachhinein dem ersten Prototypen von HTTP verliehen wurde. 1.0 ist die erste Versionsnummer, die in einem RFC veröffentlich wurde. Allerdings wurde die Version 1.0 von den Autoren noch nicht als offizieller Standard gesehen, wie das folgendes Zitat zeigt. \begin{quote} This Memo does not specify an Internet standard of any kind. \cite{Berners-Lee1996} \end{quote} Erst HTTP 1.1 bezeichnen die Autoren des RFCs als \begin{quote}[...] internet standards track protocol for the internet community. \cite{Fielding1999} \end{quote} HTTP 0.9 ist ein Protokoll mit dem man Datenpakete über eine TCP Verbindung verschicken kann. Das Protokoll definiert welche Form die Datenpakete haben müssen. Das Protokoll unterscheidet zwei Arten von Datenpaketen: Anfragen und Antworten. Anfragen sind dazu da, um Dateien anzufragen. Sie werden vom Client an den Server geschickt. Antworten enthalten bei einer erfolgreichen Anfrage die angefragte Datei. Um eine HTML Datei zu empfangen, muss ein Client über die TCP-Verbindung eine Anfrage schicken. In der Anfrage wird die Datei, die übertragen werden soll spezifiziert. Es folgt ein Beispiel für eine Anfrage.
\begin{verbatim}
GET /index.html
\end{verbatim}
Die Methode in dieser Anfrage ist GET. Sie bedeutet, dass eine Datei vom Server zum Client übertragen werden soll. Der Server, die IP-Adresse und der Port müssen in HTTP-Anfragen nicht spezifiziert werden. Das ist Aufgabe der unterliegenden Protokolle IP und TCP. Die GET-Methode bekommt als Parameter die Adresse der Datei, die übertragen werden soll. Die Adresse bezieht sich auf das Dateisystem des Servers. In obigem Beispiel geht der Client davon aus, dass die angefragte Datei auf der obersten Ebene der Dateihierarchie liegt.\newline
Wenn eine Anfrage bei einem Server ankommt, muss er entscheiden, ob er die angeforderte HTML-Datei verschicken kann. Wenn er das kann ist die Antwort in der Protokollversion 0.9 die angefragte HTML-Datei. Aufgrund des Anfrage-Antwort-Mechanismus, den HTTP nutzt, wird es der Gruppe der Request-Response-Protokolle zugeordnet. Die Codierung für die Protokollversionen 0.9, 1.0 und 1.1 ist ASCII. Das ist so, weil die Devise bei der Entwicklung von HTTP 0.9 war möglichst wenig über die Computer, die über das Protokoll kommunizieren sollten anzunehmen: \begin{quote} The least common denominator we could assume among all different types of computers was that they all head some sort of keyboard input device, and they all could produce ASCII (plain text) characters. \cite{Berners-Lee1999} \end{quote} Die HTTP Version 0.9 eignet sich gut um den Request-Response-Charakter von HTTP anschaulich darzustellen, weil die Anfragen und Antworten sehr kompakt sind. Das ist bei den Versionen mit höheren Versionsnummern nicht so. Es folgt ein Beispiel für eine Antwort, die mit HTTP 0.9 konform ist.
\begin{verbatim}
<html>
<p> Das ist eine HTML-Seite.<\p>
</html>
\end{verbatim}
\section{Methoden}
Der wichtigste Bestandteil eines HTTP-Pakets ist die Methode. Ein Datenaustausch zwischen Client und Server geht normalerweise vom Client aus. Der schickt ein Paket, dass eine Methode enthält. Sie sagt dem Server was der Client will. Weiter oben wurde die GET-Methode beschrieben. 

\section{Header}
Während man mit dem HTTP-Protokoll in der Version 0.9 nur HTML-Dateien verschicken konnte, ist es ab der Protokollversion 1.0 auch möglich andere Dateitypen zu verschicken. Das Protokoll ermöglicht es, dass ein Paket die Information enthält, welchen Dateityp das Paket verpackt. Diese Metadaten sind im Header. Der Header eines Pakets besteht aus einem oder mehreren Headerfeldern. Die Aufteilung in Metadaten und Pakethinhalt führt dazu, dass ein Antwort-Paket nach Protokollversion 1.0 eine andere Struktur hat als ein Paket nach Protokollversion 0.9. Während in Version 0.9 nur eine HTML-Datei enthalten ist, kann ein Antwort-Paket ab Version 1.0 auch einen Header enthalten, der den Inhalt des Paketes beschreibt. Die Kombination von Metadaten und Inhalt heißt Entity. Den Inhalt bezeichnet man auch als Payload oder Entity-Body.
\begin{quote}   entity \linebreak

       A particular representation or rendition of a data resource, or
       reply from a service resource, that may be enclosed within a
       request or response message. An entity consists of
       metainformation in the form of entity headers and content in the
       form of an entity body. \cite{Berners-Lee1996} \end{quote} \
   Es folgt ein Beispiel für ein Headerfeld.
\begin{verbatim}
Content-Type: text/html; charset=UTF-8
\end{verbatim}
Der oben gezeigte Header ist dafür da, um den Datentyp eines Paketinhalts anzuzeigen. Wie man oben sieht, sind Header Paare aus Name und Wert, die durch einen Doppelpunkt getrennt sind. [Hier kann ich noch etwas über MIME schreiben.] \linebreak Es gibt verschiedene Arten von Headern. Sie erfüllen unterschiedliche Funktionen. Sie sind in drei Gruppen eingeteilt: Allgemeine Header, Anfrage-Header, Antwort-Header und Entity-Header. Entity-Header beschreiben den Entity-Body. Anfrage Header dürfen nur in Anfragen benutzt werden, Antwort-Header nur in Antworten.  Allgemeine Headerfelder sind in Anfragen und Antworten erlaubt. Ein Beispiel für ein allgemeines Headerfeld ist das Connection-Headerfeld: 

\begin{verbatim}
Connection: close
\end{verbatim}
Wenn dieses Headerfeld in einer Anfrage mitgeschickt wird, sagt es aus, dass die Verbindung nach Bearbeitung der Anfrage geschlossen werden kann. 
\section{Flexibilität und Erweiterbarkeit}
Zuerst war das World Wide Web nur zum Austausch von schriftlichen Dokumenten gedacht. Im Laufe der Zeit enstand der Wunsch auch Dateien anderer Formate im World Wide Web zu veröffentlichen. Zum Beispiel Bilder. Das HTTP-Protokoll in der Version 0.9 unterstützt das nicht. Ab der Version 1.0 ist es möglich andere Dateiformate außer HTML zu verschicken. Das spiegelt sich auch in der Beschreibung des Protokolls wieder. So wird HTTP im RFC 1945 als "protocol for distributed, collaborative, hypermedia information systems bezeichnet." Der Unterschied zwischen Hypermedia und Hypertext ist, dass Hypermedia neben Text auch Bilder und Videos enthält.
Bei der Entwicklung der Protokollversion 1.0 wurde darauf geachtet, dass das Protokoll flexibel ist. Flexibilität wird durch die Möglichkeit neue Methoden und Header hinzuzufügen erreicht. 
\section{Performanz}
In HTTP/2 gibt es Multiplexing.\cite{Belshe2015}
\section{Cookies}
\section{stateless}
\bibliographystyle{acm}
\bibliography{literature}


\end{document}

              
 

